\documentclass{report}

\usepackage{hyperref}

\newcommand{\thetitle}{The Kalvaszti Language}
\newcommand{\theauthor}{Owen Bechtel}

\title{\thetitle}
\author{\theauthor}
\hypersetup{
  pdftitle={\thetitle},
  pdfauthor={\theauthor},
  colorlinks=true,
  linkcolor=blue,
}

\begin{document}
\maketitle

\chapter{Introduction: Geography, History, and Culture}

Kalvaszti is a language spoken by the fictional Kal people, who inhabit the islands of Kalnot and various recently-established overseas colonies. The universe of the Kal, known in Kalvaszti as Hom, is cosmologically very different from our own. However, the people that inhabit it are, in appearance and overall behavior, identical to Earth humans. There are no elves, dwarves, goblins, or aliens. There are, however, gods and angels, as well as spirits of varying moral character; these interact with humans only rarely. There is no ``magic system,'' although supernatural things have been known to happen from time to time, and there are a few objects scattered throughout the world with peculiar properties. The Kal fight with guns and cannons, although this technology has not yet spread to much of the world. Men travel by horse and by sail. A Kal mathematician recently computed the value of $\sum_{n = 1}^\infty 1/n^2$, solving a century-old problem. Men in Kalnot and on the nearby mainland have begun to develop the very first steam engines.  

\section{Cosmology and Theology}

Hom, the universe in which the Kal and many other peoples reside, consists of a flat, disk-shaped world, as well as a number of heavenly lights. 

\subsection{Suns, Moons, and Ether Curtains}

The latter include two suns and two moons, which orbit above the world in a circular manner such that only one of them is visible at any time, except for a brief period at sunrise and sunset when a sun and a moon are both dimly visible. To prevent the sun on the opposite side of the world from being visible, there are four ether curtains: walls of light-blocking ether that rotate along with the suns and moons and divide the world into four quadrants. The ether curtains are just wide enough to prevent all light from passing through them, so when sunrise is halfway over, and one is standing in the middle of an ether curtain, the sun and the moon are both half as bright as usual. They also appear red-tinted at these hours, because light-blocking ether is more transparent to red light than to blue light. 

% include diagram

The sky is dotted with several thousand stars of varying luminosities. These orbit in the same direction as the suns and the moons (counterclockwise), but travel very slightly slower. Like on Earth, they are only visible during the night due to the sun's much greater brightness. About once every century, a new star appears in the sky, or an existing one disappears. It is important to remember that suns, moons, and stars are not of the same nature in Hom as they are in our universe. They are not nearly as large, and are much closer to the world.

Hom has at least six cosmological cycles:
\begin{itemize}
  \item The quotidian cycle: the alternation between day and night. The period of this cycle is known as a ``day,'' and is approximately one Earth day.
  \item The orbital cycle: the time it takes each sun and moon to orbit the earth. Since there are two suns and two moons, this is exactly two days.
  \item The stellar cycle: the time it takes each star to orbit the earth. This is exactly 12673/6336 days --- about 2.00016 days. After 12673 days, or 6336 stellar cycles --- about 35 years --- the stars will have orbited an integer number of times, and the suns and moons will have orbited a half-integer number of times. So the same set of stars will be visible again, but under opposite moons.  
  \item The annual cycle: % 363 days
  \item The lunar cycle: The moon % 35 days
  \item The tidal cycle: %221/?
\end{itemize}

% orbital: 2
% stellar: 17, 
% annual: 3, 11
% lunar: 5, 7
% tidal: 13, 17 

\section{Geography}

\section{Characteristics of the Kal People}

The Kal people are basically white people with curly brown hair and light brown eyes. They have light skin, as well as facial features associated on Earth with western Eurasia and Europe in particular, such as relatively narrow noses. Their skin color is due mostly to the cool climate of Kalnot and Sedem --- not because of natural selection, but because the gods designed the races to be well-adapted to their respective environments. (Although natural selection has of course taken place since then.) The people of Hom do not all share a common ancestor.

\section{History of Kalnot}

\chapter{Phonology and Sandhi}

\chapter{Nouns}

\chapter{Verbs}

\chapter{Numbers and Pronouns}

\chapter{Sentences}

% \chapter{Derivation}
% \chapter{Additional notes}

\appendix
\chapter{Proto-Kalvaszti}

\chapter{Dictionary}
% Swadesh-207 list

\end{document}