\documentclass{article}

\usepackage{hyperref}
\usepackage{tipa}

\newcommand{\w}{\super w}

\newcommand{\thetitle}{The Universe of the Kal People}
\newcommand{\theauthor}{Owen Bechtel}

\title{\thetitle}
\author{\theauthor}
\hypersetup{
pdftitle={\thetitle},
pdfauthor={\theauthor}}

\begin{document}
\maketitle

\section{Introduction}

Hom, the world herein described, is inhabited by many peoples, including the Kal people, who speak the Kalvaszti language. The term \textit{Hom} comes from the Kalvaszti word \textipa{/h\w om/} --- other peoples have different names for the world.

Hom is cosmologically very different from our own universe. For one thing, the world is flat and circular, rather than spherical. But the people that inhabit it are, in appearance and biology, identical to Earth humans. There are no elves, dwarves, goblins, or aliens. There are, however, gods and angels, as well as spirits of varying moral character; these interact with humans only rarely. There is no ``magic system,'' although supernatural things have been known to happen from time to time, and there are a few objects scattered throughout the world with peculiar properties. The Kal fight with guns and cannons, although this technology has not yet spread to much of the world. Men travel by horse and by sail. A Kal mathematician recently computed the value of $\sum_{n = 1}^\infty 1/n^2$, solving a century-old problem. Men in Kalnot and on the nearby mainland have begun to develop the very first steam engines.

The present age is an age of expansion for the Kal people. In the last few centuries, they have settled numerous previously-uninhabited islands throughout the ocean. About one hundred years ago, they first made contact with the vast continent of Terba across the sea, inhabited by many peoples formerly unknown to them. They are beginning to explore the interior, and are meeting some hostility from the native pastoralists.  

\section{Cosmology and Theology}

\subsection{Suns, Moons, and Ether Curtains}

There are two suns and two moons, which travel in circles above the world. To an observer on the ground, only one of these is visible at once, except at sunrise and sunset, when a sun and a moon are both dimly visible. To prevent the sun on the opposite side of the world from begin visible, there are four curtains of dark ether that rotate along with the suns and moons and divide the world into quadrants. Dark ether is a substance that interacts only with light and not with matter. A wider and/or denser curtain of dark ether will allow less light to pass through it. The ether curtains in Hom are just wide enough to prevent essentially all light from passing through them, so that the suns are not visible during the night. Sunset and sunrise --- the periods in which one is inside an ether curtain --- each last about 1/50 of a day, or 30 minutes. The length of a day on Hom is not significantly different from that of a day on Earth. ``Day'' in this context refers to a single day-night cycle, or half a rotation.

% include diagram

Like our own moon, the moons of Hom go through a cycle, but instead of changing shape, they simply change in brightness while remaining circular. Hom's lunar cycle has a period of about 40 days. At their minimum brightness, the moons emit no light at all. At their maximum, they are as bright as a full moon on Earth.

\subsection{Stars}

The sky contains around 100,000 stars of varying luminosities. 90--95\% of these stars are not visible with the naked eye, and can only be seen with the use of a telescope or a similar device. Like on Earth, the stars are only visible at night due to the sun's much greater brightness. The stars orbit in the same direction as the suns and the moons (counterclockwise), but travel very slightly slower. Ignoring this difference in speed, the night sky alternates between two different sets of stars, as it takes two days for the sky to return to its original position. The tiny difference in speed between the stars and the moons has the effect that, after about 40 years, the moons will have rotated by one half turn relative to the stars, so the same set of stars will be visible once again, but under the opposite moon.

About once per century, a new star appears in the sky, or an existing one disappears. The folk religions of Hom have a variety of explanations for why this occurs, the most common being that a star is added when a great man dies, and removed when a great man is born.

It is important to remember that suns, moons, and stars are not of the same nature in Hom as they are in our universe. They are simply glowing orbs floating in the sky.

\subsection{Tides}

\subsection{Seasons}

\subsection{Speculation}

\section{Geography}

\section{Characteristics of the Kal People}

\section{History of Kalnot}

\end{document}